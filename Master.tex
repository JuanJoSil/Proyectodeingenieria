\documentclass[12pt,letterpaper]{book}

%Paquetes esenciales
\usepackage[top=3 cm,bottom=3.cm,right=3cm,left=3cm]{geometry} %Para definir y diseñar los margenes de la pagina
\usepackage{pdfpages}
\usepackage[utf8]{inputenc} %Permite al usuario introducir caracteres acentuados directamente desde el teclado (latin1, utf8, koi8r) Tambien se puede usar ambos (\usepackage[T1]{fontenc} y \usepackage[latin1]{inputenc}) La combinación inputenc-fontenc parece lento y engorroso, pero es seguro. Esta combinacion s útil para documentos en inglés; tan pronto como utilices cualquier caracter fuera del codigo ASCII, necesitarás estos paquetes.
\usepackage{graphicx} %Para importar las gráficas
\usepackage{makeidx} %Crea un indice con \makeindex
\usepackage[english]{babel} %Permite escribir lenguaje en español (ñ)
\usepackage{xcolor} %Soporta un gran numero de colores y es mas flexible que color
\usepackage{hyperref} %Maneja  los comandos de referencias cruzadas para producir enlaces de hipertexto en el documento, útil cuando el formato resultante es PDF y los hipervínculos pueden seguirse. [colorlinks] escribe ()citas, indice, figuras, tablas, pie de notas) en rojo sin recuadros, [colorlinks=false] u omitiendolo escribe en negro con cuadros rojos,  [colorlinks=false, linktocpage=true] escribe en negro con cuadros rojos unicamente las paginas y citas.
\usepackage{cite} %Comprime listas de 3 o mas citas consecutivas
\usepackage{amsmath, amsthm, amssymb, amsfonts,amsbsy,amstext,amscd,amsxtra,amsopn} %Debido a que amssymb carga internamente a amsfonts, es suficiente cargar el primero.
\usepackage{epstopdf} %Cuando usas imagenes con extension eps, pueden convertilas a pdf
\usepackage[titletoc,toc,page]{appendix}
\usepackage{fancyhdr} %para modificar cabeceras y pies de páginas
\usepackage{float} %Para adecuar las tablas y figuras
%\usepackage{subcaption} %Los títulos de figuras, tablas, subfiguras y subtablas en LaTeX se pueden personalizar de varias maneras mediante el paquete de \usepackage{caption} y \usepackage{subcaption}
%\usepackage{tikz} %Para crear elementos graficos, lineas, puntos, curvas, circulos, rectangulos, etc (https://www.sharelatex.com/learn/TikZ_package)
%\usepackage{picture}
\usepackage{multicol}
\usepackage{subfigure}
\usepackage{dsfont}
\usepackage{bbm}
\usepackage{bbold}
\usepackage{microtype}
\usepackage{physics}         %necesito physics.sty
\usepackage{braket} %escribir brakets en latex
\usepackage[mathscr]{euscript}
\usepackage{bbm} %1 \mathbbm{1}
%\spanishdecimal{.}
\usepackage{empheq}
\usepackage{cancel}
\usepackage{tensor}
\usepackage{comment}
\newcommand{\harpoon}{\overset{\rightharpoonup}}

\usepackage[pages=some]{background}%Para colocar una imagen de fondo (http://texdoc.net/texmf-dist/doc/latex/background/background.pdf)
% % % % % % % % % % % % % % % % % % % % % % % % % % % % % % % % % % % % % % % % % % % % % % % % % % % % % % % % % % % % % % % % % % % % % % % % %
\floatplacement{figure}{H} %Comando junto con \usepackage{float} para hacer que las imagenes se coloquen justo donde se quiere.

\hypersetup{ hidelinks,} %Con éste y \usepackage{hyperref} escribe las citas en negro y sin cuadros

\pagestyle{fancyplain} %Con éste y \usepackage{fancyhdr} modificas cabeceras y pie de notas
%\renewcommand{\chaptermark}[1]{%
%\markboth{\thechapter.\ #1}{}}
\fancyhf{}
%\fancyhead[LE,RO]{\thepage} %Numero de pagina en la cabecera lado izquierdo paginas par
\fancyhead[RE]{\normalsize\nouppercase{\leftmark}} %Capitulo a la derecha R en pags. pares
\fancyhead[LO]{\normalsize\nouppercase{\rightmark}} %Seccion a la izquierda en pags. impares
%\fancyfoot[CE,CO]{\leftmark} %Capitulo centrado en ambas paginas
\fancyfoot[LE,RO]{\thepage} %Numero de pagina en el pie de pagina L(left field) E(even page) R(right field) O(odd page) C(center field) H(headers) F(footer)

\renewcommand{\headrulewidth}{0.01pt} %gruero de la linea en la cabecera
\renewcommand{\footrulewidth}{0pt} %gruero de la linea en el pie de pagina
% % % % % % % % % %Comandos sugeridos por Champi % % % % % % % % % % % % % %
\newcommand{\fullref}[1]{\ref{#1} de la p\'{a}gina \pageref{#1}}


\makeindex

\begin{document}

\frontmatter %Frontmatter
        \backgroundsetup{
scale=1,
color=white,
opacity=1,
angle=0,
contents={%
  \includegraphics[width=\paperwidth,height=\paperheight]{portada.pdf}
  }%
} %incluir una imagen de fondo
\begin{titlepage}
\BgThispage %Fondo solamente en esta página
%   \parbox[s][22 cm][c]{18.145 cm}{
    \begin{center}
		~\\[5cm]
        \Large
        \textbf{}\\[0.8cm] %Titulo de la tesis
        \large
        Por\\[0.8cm]
        \textbf{Juan José Silva Cuevas}\\ %Nombre del tesista
%        \\[0.8cm] %Adscripcion donde obtuvo el último grado académico
        Proyecto de Ingeniería Física\\[0.8cm]
       % \textbf{Maestro en Ciencias en la especialidad en Óptica}\\[0.8cm] %Grado académico que se obtieneen el\\[0.8cm]
		\textbf{Tecnol\'ogico de Monterrey}\\ %INAOE
   %     Agosto 2016\\ %Fecha de entrega, mes y año
        Monterrey, Nuevo Le\'on\\[0.8cm] %Lugar del INAOE
        Supervisada por:\\[0.8cm]
		\textbf{Dr. Blas Manuel Rodríguez Lara }\\ %Director y co-director (no más de 2)
        Profesor Investigador, \\
        Escuela Nacional de Ingenier\'ia y Ciencias,\\
        Tecnol\'ogico de Monterrey, campus Monterrey.\\[2.1cm]
        \copyright Instituto Tecnológico y de Estudios Superiores de Monterrey 2017\\ %Derechos reservados al INAOE
        Derechos reservados\\
        El autor otorga al ITESM el permiso de reproducir y\\
        distribuir copias de esta tesis en su totalidad o en partes
    \end{center}
%    }
\end{titlepage}

		%%%%%%%%%%%%%%%%%%%%%%%%%%%%%%%%%%%%%%%%%%%%%%%%%%%%%%%%%%%%
% Abstract
%%%%%%%%%%%%%%%%%%%%%%%%%%%%%%%%%%%%%%%%%%%%%%%%%%%%%%%%%%%%
\chapter*{Abstract}
\addcontentsline{toc}{chapter}{\ Abstract}


%%%%%%%%%%%%%%%%%%%%%%%%%%%%%%%%%%%%%%%%%%%%%%%%%%%%%%%%%%%%
% Resumen
%%%%%%%%%%%%%%%%%%%%%%%%%%%%%%%%%%%%%%%%%%%%%%%%%%%%%%%%%%%%
\chapter*{Resumen}
\addcontentsline{toc}{chapter}{\ Resumen}

		%%%%%%%%%%%%%%%%%%%%%%%%%%%%%%%%%%%%%%%%%%%%%%%%%%%%%%%%%%%%
% Agradecimientos
%%%%%%%%%%%%%%%%%%%%%%%%%%%%%%%%%%%%%%%%%%%%%%%%%%%%%%%%%%%%
\chapter*{Agradecimientos}
\addcontentsline{toc}{chapter}{\ Agradecimientos}



\vspace{0.8cm}
\begin{flushright}
%CINTHIA HUERTA ALDERETE\\
%Instituto Nacional de Astrof\'isica, Óptica y Electr\'onica\\
%Coordinaci\'on de \'Optica\\
%Puebla, Pue, 11 de agosto de 2016
\end{flushright}


%%%%%%%%%%%%%%%%%%%%%%%%%%%%%%%%%%%%%%%%%%%%%%%%%%%%%%%%%%%%
% Antecedentes y motivación de la tesis
%%%%%%%%%%%%%%%%%%%%%%%%%%%%%%%%%%%%%%%%%%%%%%%%%%%%%%%%%%%%
\chapter*{Dedicatoria}
\addcontentsline{toc}{chapter}{\ Dedicatoria}

\begin{flushright}
\Huge{
\textbf{\emph{A mi familia. }}
}
\end{flushright}



    \tableofcontents
    %\listoffigures


\mainmatter
%	\part{Introducción}
	    %%%%%%%%%%%%%%%%%%%%%%%%%%%%%%%%%%%%%%%%%%%%%%%%%%%%%%%%%%%%
% Antecedentes y motivación de la tesis
%%%%%%%%%%%%%%%%%%%%%%%%%%%%%%%%%%%%%%%%%%%%%%%%%%%%%%%%%%%%
\chapter*{Introduction}
\addcontentsline{toc}{chapter}{\ Introduction}


	
%	\part{AcDebil}
        %%%%%%%%%%%%%%%%%%%%%%%%%%%%%%%%%%%%%%%%%%%%%%%%%%%%%%%%%%%%
% Motivación de la tesis
%%%%%%%%%%%%%%%%%%%%%%%%%%%%%%%%%%%%%%%%%%%%%%%%%%%%%%%%%%%%
\chapter{}
%\addcontentsline{toc}{chapter}{\ Thorough Derivation of the Rabi Equation}

%%%%%%%%%%%%%%%%%%%%%%%%%%%%%%%%%%%%%%%%%%%%%%%%



        %%%%%%%%%%%%%%%%%%%%%%%%%%%%%%%%%%%%%%%%%%%%%%%%%%%%%%%%%%%%
% Motivación de la tesis
%%%%%%%%%%%%%%%%%%%%%%%%%%%%%%%%%%%%%%%%%%%%%%%%%%%%%%%%%%%%
\chapter{}
%\addcontentsline{toc}{chapter}{\ Thorough Derivation of the Rabi Equation}

%%%%%%%%%%%%%%%%%%%%%%%%%%%%%%%%%%%%%%%%%%%%%%%%


        %%%%%%%%%%%%%%%%%%%%%%%%%%%%%%%%%%%%%%%%%%%%%%%%%%%%%%%%%%%%
% Motivación de la tesis
%%%%%%%%%%%%%%%%%%%%%%%%%%%%%%%%%%%%%%%%%%%%%%%%%%%%%%%%%%%%
\chapter{}


        %%%%%%%%%%%%%%%%%%%%%%%%%%%%%%%%%%%%%%%%%%%%%%%%%%%%%%%%%%%%
% Motivación de la tesis
%%%%%%%%%%%%%%%%%%%%%%%%%%%%%%%%%%%%%%%%%%%%%%%%%%%%%%%%%%%%
\chapter{}



\appendix

%    \part{Apéndices}
		%%%%%%%%%%%%%%%%%%%%%%%%%%%%%%%%%%%%%%%%%%%%%%%%%%%%%%%%%%%%
% Motivación de la tesis
%%%%%%%%%%%%%%%%%%%%%%%%%%%%%%%%%%%%%%%%%%%%%%%%%%%%%%%%%%%%
%\chapter{Simulaci\'on cu\'antica con iones atrapados}


%%%%%%%%%%%%%%%%%%%%%%%%%%%%%%%%%%%%%%%%%%%%%%%%%%%%%%%%%%%%
% Trapped-ion quantum simulation
%%%%%%%%%%%%%%%%%%%%%%%%%%%%%%%%%%%%%%%%%%%%%%%%%%%%%%%%%%%%




%    \part{Publicaciones}
	%	%%%%%%%%%%%%%%%%%%%%%%%%%%%%%%%%%%%%%%%%%%%%%%%%%%%%%%%%%%%%
% Publicaciones
%%%%%%%%%%%%%%%%%%%%%%%%%%%%%%%%%%%%%%%%%%%%%%%%%%%%%%%%%%%%
\chapter{Publicaciones}

%\includepdf[pages={1-7}]{Filename.pdf}

\backmatter

	\bibliographystyle{ieeetr}	
	\bibliography{ref}


\end{document} 
*.tdo
JJSC.pdf
